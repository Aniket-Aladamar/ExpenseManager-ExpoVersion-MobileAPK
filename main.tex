\documentclass[12pt,a4paper]{report}
\usepackage[utf8]{inputenc}
\usepackage[margin=1in]{geometry}
\usepackage{setspace}
\usepackage{graphicx}
\usepackage{hyperref}
\usepackage{titlesec}
\usepackage{float}
\usepackage{longtable}
\usepackage{booktabs}
\usepackage{array}
\usepackage{xcolor}

% Double spacing
\linespread{1.5}

% Chapter title formatting
\titleformat{\chapter}[display]
  {\normalfont\huge\bfseries\centering}{\chaptername\ \thechapter}{20pt}{\Huge}
\titlespacing*{\chapter}{0pt}{50pt}{40pt}

\begin{document}

% Title page
\begin{titlepage}
    \centering
    % Example: University logo can go here
    \includegraphics[width=0.3\textwidth]{media/Unilogo.png}\\[0.5cm]
    {\Large\bfseries RV University}\\[0.2cm]
    {\Large\bfseries School of Computer Science and Engineering}\\[1cm]

    {\Huge\bfseries SmartSpend: Personal Expense Manager Mobile Application}\\[1.5cm]
    {\Large B.Tech Computer Science}\\[0.2cm]
    {\large Submitted by}\\[0.5cm]
    Aniket Aladamar -- 1RVU22CSE026\\[0.2cm]
    Deekshith R Prabhu -- 1RVU22CSE046\\[1cm]

    {\large Under the Guidance of}\\[1 cm]
    {\Large Prof. Rajeshwari Adrakatti}\\[0.5cm]
    Professor, School of Computer Science and Engineering\\
    RV University, Bengaluru--560059\\[1.5cm]
    {\large Academic Year 2025-2026}
\end{titlepage}

% Certificate
\chapter*{Certificate}
\addcontentsline{toc}{chapter}{Certificate}

\vspace{1cm}

Certified that the project work titled \textbf{`SmartSpend: Personal Expense Manager Mobile Application'} is carried out by \textbf{Aniket Aladamar} (1RVU22CSE026) and \textbf{Deekshith R Prabhu} (1RVU22CSE046), students of RV University, Bengaluru, \textbf{B.Tech in the School of Computer Science and Engineering} during the year 2025-2026. It is certified that all corrections/suggestions indicated for the Internal Assessment have been incorporated in the project report. The project report has been approved as it satisfies the academic requirements in respect of project work prescribed by the institution.

\vspace{3cm}

\textbf{Signature of Guide}

\vspace{3cm}

\textbf{External Viva:}
\vspace{1cm}

\begin{tabular}{p{6cm} p{6cm}}
\textbf{Name of Examiners} & \textbf{Signature with Date} \\
& \\
\textbf{1} & \\
& \\
\textbf{2} & \\
& \\
\end{tabular}

\vspace{1.5cm}

% Declaration
\chapter*{Declaration}
\addcontentsline{toc}{chapter}{Declaration}

\vspace{1cm}

We, \textbf{Aniket Aladamar} (1RVU22CSE026) and \textbf{Deekshith R Prabhu} (1RVU22CSE046), students of seventh semester B.Tech, SoCSE, RV University, Bengaluru, hereby declare that the project titled \textcolor{blue}{SmartSpend: Personal Expense Manager Mobile Application} has been carried out by us and submitted in partial fulfillment of \textbf{Bachelor of Technology in School of Computer Science and Engineering} during the year 2025-26.

\vspace{1cm}

Further, we declare that the content of the report has not been submitted previously by anybody or to any other university.

\vspace{1cm}

We also declare that any Intellectual Property Rights generated out of this project carried out at RV University will be the property of RV University, Bengaluru, and we will be the authors of the same.

\vspace{2cm}

\textbf{Place:} Bengaluru \\
\textbf{Date:}

\vspace{2cm}

\begin{tabular}{l l}
\textbf{Name} \hspace{10cm} & \textbf{Signature} \\
1. \textbf{Aniket Aladamar (1RVU22CSE026)} & \\
2. \textbf{Deekshith R Prabhu (1RVU22CSE046)} & \\
\end{tabular}

% Acknowledgement
\chapter*{Acknowledgement}
\addcontentsline{toc}{chapter}{Acknowledgement}

It is a great pleasure for us to acknowledge the assistance and support of many individuals who have been responsible for the successful completion of this project.\\[0.5cm]

First, we take this opportunity to express our sincere gratitude to the School of Computer Science and Engineering, RV University, for providing us with a great opportunity to pursue our bachelor's degree in this institution.\\[0.5cm]

A special thanks to our Program Director, \textbf{Dr. Sudhakar K. N} and Dean -- \textbf{Dr. Shobha G}, for their continuous support and providing the necessary facilities with guidance to carry out this project work.\\[0.5cm]

We would like to thank our guide, \textbf{Prof. Rajeshwari Adrakatti}, School of Computer Science and Engineering, RV University, for sparing her valuable time to extend help in every step of our project work, which paved the way for smooth progress and fruitful culmination of the project.\\[0.5cm]

We are also grateful to our family and friends who provided us with every requirement throughout the course.\\[0.5cm]

We would like to thank one and all who directly or indirectly helped us in the project work.

\vspace{3cm}

\begin{flushright}
\textit{Signature of Students} \hspace{3cm} USN:
\end{flushright}

% Abstract
\chapter*{Abstract}
\addcontentsline{toc}{chapter}{Abstract}

Managing personal and professional expenses has become increasingly complex for students and working professionals who juggle multiple payment modes, categories, and reimbursements. Manual tracking using spreadsheets or paper-based methods often leads to missing receipts, poor visibility into spending patterns, and difficulty in preparing reimbursement claims. This project presents \textbf{SmartSpend}, a cross-platform mobile application built using React Native and Expo that enables users to capture, categorize, and analyze their day-to-day expenses in a structured and visually intuitive manner.

SmartSpend integrates with Firebase for secure authentication and cloud-backed data storage, allowing users to log in, add expenses, and access their data across devices. The application supports detailed expense entry, including vendor, amount, date, category, expense type (personal or business), GST details, and optional receipt attachments captured via camera, gallery, or document upload. A dashboard screen aggregates this data to provide monthly summaries, category-wise breakdowns using charts, and quick access to recent expenses. Additional screens support listing, filtering, and managing expenses, along with basic profile and reporting views.

The system has been designed with usability, responsiveness, and real-time synchronization in mind, leveraging Firestore as the backend database and React Context for state management. Through this work, we demonstrate how modern mobile technologies can be combined to deliver a practical and extendable personal finance tool that improves financial awareness and simplifies expense management for end users.

\textbf{Keywords:} Mobile application, personal finance, expense tracking, React Native, Expo, Firebase Authentication, Cloud Firestore, dashboard analytics.

% Table of Contents, List of Tables, List of Figures
\tableofcontents
\listoftables
\listoffigures

%------------------ MAIN CHAPTERS ------------------

\chapter{Introduction}
\section{General Introduction}
Managing money effectively has become a critical skill for students, young professionals, and working adults who handle multiple payment methods, subscriptions, and irregular expenses. Many individuals still rely on informal techniques such as paper notes, simple spreadsheets, or bank SMS alerts to keep track of their spending. These methods often lead to incomplete records, lost receipts, difficulty in distinguishing between personal and reimbursable expenses, and a general lack of insight into where money is going each month.

Smartphones and cloud platforms provide an opportunity to address these challenges by offering always-available, user-friendly tools for expense tracking and basic financial analysis. A well-designed mobile application can help users record expenses at the point of transaction, categorize spending, attach receipts in digital form, and visualize spending patterns over time. Such an application reduces manual effort, improves accuracy, and supports better budgeting and financial decision-making.

This project focuses on the design and development of \textbf{SmartSpend}, a cross-platform mobile expense manager built using React Native and Expo. The application integrates with Firebase for user authentication and secure, cloud-based data storage. Through features such as detailed expense entry, category selection, receipt capture, dashboard analytics, and separation of personal and business expenses, SmartSpend aims to provide a practical and extendable solution for everyday expense management.

\section{Literature Survey}
The rapid adoption of smartphones has led to the emergence of many personal finance and expense-tracking applications. These tools differ in terms of target users, feature sets, and underlying technologies. Some focus on simple manual entry of expenses, while others provide bank integrations, budgeting tools, or advanced analytics. Common themes in the literature include the importance of usability, data security, cross-platform availability, and meaningful visualizations for helping users understand their spending behaviour.

Several studies and industry reports highlight that mobile apps which offer immediate feedback, category-wise summaries, and notifications tend to improve users' financial awareness and adherence to budgets. Research on human-computer interaction also emphasizes the need for minimal friction at the point of data entry, such as quick forms, intelligent defaults, and support for scanning receipts instead of typing all details manually. Cloud-backed storage and authentication frameworks like Firebase have become popular choices for small teams and academic projects because they simplify user management and real-time data synchronization.

The design of SmartSpend is informed by these findings. The application adopts a clean, card-based interface, category-based tagging of expenses, and dashboard charts that summarize spending. By using React Native and Expo, the project targets both Android and iOS with a single codebase, while Firebase Authentication and Cloud Firestore provide secure and scalable backend services. This combination aligns with modern best practices observed in recent academic prototypes and commercial personal finance applications.

\section{Problem Statement}
Students and young professionals often struggle to maintain an accurate and up-to-date record of their daily expenses. Existing solutions such as notebooks, spreadsheets, or generic finance apps may not fully address their needs, especially when it comes to separating personal and reimbursable expenses, attaching receipts for later reference, and getting a clear monthly overview. As a result, users face difficulties in answering basic questions such as how much they spent in a given category, which expenses are pending reimbursement, or whether they are staying within an informal budget.

There is a need for a focused, easy-to-use mobile application that supports structured expense entry, organized storage of receipts, and simple visual analytics, while remaining lightweight enough for everyday use. The solution should leverage cloud services to keep data synchronized across devices and ensure that users do not lose their records when they change or reset their phones.

\section{Objectives}
\begin{itemize}
 \item To design and develop a cross-platform mobile application for recording personal and business expenses.
 \item To implement secure user authentication and cloud-based storage using Firebase.
 \item To enable users to categorize expenses and attach receipts via camera, gallery, or file upload.
 \item To provide a dashboard with monthly summaries, category-wise breakdowns, and recent expense lists.
 \item To ensure a simple, intuitive user interface suitable for students and working professionals.
\end{itemize}

\chapter{System Design}
\section{System Architecture}
\begin{figure}[H]
\centering
\includegraphics[width=0.95\textwidth]{system-architecture.jpeg} % Insert your actual diagram
\caption{High-level system architecture of SmartSpend mobile application}
\label{fig:system-architecture} 
\end{figure}

The proposed system follows a client--backend architecture in which a React Native mobile application communicates with Firebase services to store and retrieve expense data securely. The overall workflow of SmartSpend is illustrated in Figure~\ref{fig:system-architecture}.

\subsection{Client Layer}
The client layer consists of a cross-platform mobile application developed using React Native and Expo. This app runs on Android (and can be extended to iOS) and provides the primary user interface for:
\begin{itemize}
    \item User authentication (login, signup) using email and password.
    \item Adding new expenses with details such as vendor, amount, date, category, type, and optional description.
    \item Capturing or uploading receipts through the device camera, gallery, or file picker.
    \item Viewing dashboard summaries, category-wise charts, and recent expense lists.
    \item Navigating between screens such as Dashboard, Add Expense, Expense List, Reports, and Profile.
\end{itemize}

The UI is built using reusable components (cards, buttons, inputs) and a centralized theme (colors, spacing, typography) to maintain a consistent look and feel across the application.

\subsection{Backend and Data Layer}
The backend for SmartSpend is implemented using Firebase services, which provide managed infrastructure for authentication and data storage without requiring a custom server.

\begin{itemize}
    \item \textbf{Firebase Authentication:} Handles user registration and login. Each authenticated user is assigned a unique identifier (UID), which is used to associate expenses with the correct account.
    \item \textbf{Cloud Firestore:} Stores expense records in a cloud-hosted NoSQL database. Each document typically includes fields such as userId, vendor, amount, date, category, type, description, GST information, and optional receipt URL.
    \item \textbf{Firebase Storage (optional):} Can be used to store receipt images in the cloud if the application is extended beyond local URIs.
\end{itemize}

The React Native app communicates with Firestore through the Firebase SDK to create, read, and query expense documents. For example, the dashboard screen queries all expenses for a given user and computes monthly totals, reimbursable amounts, and category-wise aggregates on the client side.

\subsection{External Libraries and Integrations}
In addition to Firebase, SmartSpend uses several third-party libraries to enhance functionality and user experience:
\begin{itemize}
    \item Form handling and validation using Formik and Yup on the Add Expense screen.
    \item React Navigation for screen transitions and stack/tab navigation.
    \item Chart libraries for rendering pie charts and other visualizations on the dashboard.
    \item Expo Image Picker and Document Picker for capturing images and selecting files from the device.
\end{itemize}

These libraries integrate seamlessly with the React Native ecosystem and reduce the amount of boilerplate code required to implement common mobile features.

\subsection{Core Workflow Concepts}
The architecture of SmartSpend follows several design principles:
\begin{enumerate}
    \item \textbf{Client-centric processing:} Most business logic, such as calculating monthly totals and category breakdowns, is performed on the device after fetching data from Firestore. This keeps the backend simple and reduces server-side complexity.
    \item \textbf{Real-time synchronization:} By using Firestore, expense data can be synchronized across devices and updated in near real time whenever a user adds or edits an expense.
    \item \textbf{Modularity and reusability:} Common UI elements and utilities are organized into shared components and helper functions, making it easier to maintain and extend the application.
    \item \textbf{Security and privacy:} User data is protected through Firebase Authentication and Firestore security rules, ensuring that each user can only access their own expense records.
\end{enumerate}

This architecture effectively combines a modern mobile front end with managed cloud services to deliver a practical, scalable, and user-friendly expense management solution.



\chapter{Software Requirements}
\section{Functional Requirements}
\begin{itemize}
 \item User registration and authentication using email and password.
 \item Add, view, and manage expenses with fields such as vendor, amount, date, category, type, and description.
 \item Capture or upload receipts using camera, gallery, or file picker.
 \item Display monthly summaries and category-wise breakdowns of expenses on the dashboard.
 \item List recent expenses and allow navigation to detailed expense views.
\end{itemize}

\section{Non-Functional Requirements}
\begin{itemize}
 \item Usable and responsive user interface on common Android devices.
 \item Reliable synchronization of data with the cloud backend.
 \item Secure storage and access control for user-specific expense data.
 \item Maintainability and extensibility of the codebase using modular components.
\end{itemize}

\section{Hardware Requirements}
\begin{itemize}
 \item Android smartphone capable of running recent versions of the Expo Go application.
 \item Development machine (laptop/desktop) with internet connectivity to run Node.js, Expo CLI, and Android emulator (optional).
\end{itemize}

\section{Software Requirements}
\begin{itemize}
    \item \textbf{Mobile App Framework:} React Native with Expo
    \item \textbf{Runtime and Tools:} Node.js, npm/yarn, Expo CLI
    \item \textbf{Backend Services:} Firebase Authentication, Cloud Firestore, Firebase Storage (optional)
    \item \textbf{Libraries:} Formik, Yup, React Navigation, charting libraries, Expo Image Picker and Document Picker
    \item \textbf{IDE/Editor:} Visual Studio Code or any suitable JavaScript/TypeScript editor
\end{itemize}



\chapter{Implementation and Testing}
\section{Implementation Details}
The SmartSpend application was implemented using React Native version 0.71+ with Expo SDK 49+. The project structure follows standard React Native conventions with separate directories for components, screens, contexts, navigation, configuration, and utilities.

\subsection{Authentication Module}
The authentication system was built using Firebase Authentication with email/password provider. A custom AuthContext was created using React Context API to manage authentication state across the application. The LoginScreen and SignupScreen components utilize Formik for form handling and Yup for validation, ensuring proper email format and password strength requirements.

\subsection{Expense Management}
The core expense tracking functionality is implemented in the AddExpenseScreen, which provides a comprehensive form with the following fields:
\begin{itemize}
    \item Vendor name (text input)
    \item Amount (numeric input with decimal support)
    \item Date (date picker)
    \item Category (dropdown selector with predefined categories: Food, Transport, Shopping, Bills, Entertainment, Healthcare, Education, Other)
    \item Type (Personal or Business toggle)
    \item GST information (optional percentage and amount)
    \item Description (multiline text area)
    \item Receipt attachment (camera, gallery, or file picker)
\end{itemize}

Expense data is stored in Cloud Firestore under a collection named `expenses`, with each document containing the user's unique ID to ensure data isolation. The ExpenseListScreen queries Firestore using the authenticated user's ID and displays expenses in a scrollable list with category icons and color-coding.

\subsection{Dashboard and Analytics}
The DashboardScreen aggregates expense data on the client side to compute:
\begin{itemize}
    \item Total expenses for the current month
    \item Total reimbursable (business) expenses
    \item Category-wise spending distribution (displayed using pie charts)
    \item Recent expense list (last 5-10 transactions)
\end{itemize}

Chart visualizations were implemented using a charting library compatible with React Native, rendering interactive pie charts for category breakdowns.

\subsection{Navigation Structure}
The application uses React Navigation with a stack navigator for authentication flows and a bottom tab navigator for the main application screens:
\begin{itemize}
    \item Dashboard (home screen with analytics)
    \item Add Expense (expense entry form)
    \item Expense List (view all expenses)
    \item Reports (additional analytics)
    \item Profile (user account information)
\end{itemize}

\section{Testing}
Testing was conducted in multiple phases to ensure application reliability and user experience.

\subsection{Unit Testing}
Individual utility functions were tested, including:
\begin{itemize}
    \item Date formatting and validation helpers
    \item Form validation rules for expense fields
    \item Category calculation and aggregation functions
\end{itemize}

\subsection{Integration Testing}
End-to-end flows were tested to verify:
\begin{itemize}
    \item User signup with valid and invalid credentials
    \item User login and session persistence
    \item Adding expenses with all field combinations
    \item Receipt attachment from camera and gallery
    \item Expense data synchronization with Firestore
    \item Dashboard updates after adding new expenses
\end{itemize}

\subsection{Usability Testing}
The application was tested on multiple Android devices with different screen sizes to ensure responsive layout and touch interactions. Feedback was collected from test users regarding:
\begin{itemize}
    \item Ease of expense entry
    \item Clarity of dashboard visualizations
    \item Navigation intuitiveness
    \item Overall user experience
\end{itemize}

Based on testing feedback, minor UI adjustments were made to improve button sizes, form field spacing, and chart readability.

\subsection{Performance Testing}
The application's performance was evaluated under various conditions:
\begin{itemize}
    \item Loading times for dashboard with 100+ expenses
    \item Firestore query response times
    \item Image upload and rendering performance
    \item App startup time on different devices
\end{itemize}

All performance metrics were within acceptable ranges for a mobile application, with dashboard load times under 2 seconds even with large datasets.


\chapter{Conclusion and Future Work}
\section*{Conclusion}
In this project, we have designed and implemented \textbf{SmartSpend}, a mobile application that helps users record, organize, and analyze their day-to-day expenses. By leveraging React Native and Expo on the client side and Firebase services on the backend, the system provides a cross-platform solution with secure authentication and cloud-backed data storage. Users can add detailed expenses, categorize them as personal or business, attach receipts, and view monthly summaries and category-wise breakdowns on an interactive dashboard.

The modular architecture, use of shared UI components, and integration of third-party libraries for forms, validation, navigation, and charting have enabled the development of a maintainable and user-friendly application. SmartSpend demonstrates how modern mobile and cloud technologies can be combined to address common financial management challenges faced by students and professionals, improving visibility into spending patterns and supporting better decision-making.




\section*{Future Work}
Several enhancements can be explored in future iterations of SmartSpend. Budgeting features and alerts could be added to allow users to set monthly limits and receive notifications when approaching or exceeding them. Optical character recognition (OCR) could be integrated to automatically read amounts and vendor names from receipt images, reducing manual data entry.

Additional improvements may include multi-currency support, export of expense data to CSV or PDF for sharing or reimbursement claims, and more advanced analytics such as trends over longer periods. Integration with bank SMS or email parsing, subject to security and privacy considerations, could further automate expense capture. These extensions would make SmartSpend an even more comprehensive personal finance assistant.


% -------------------- REFERENCES (APA Style) --------------------
\clearpage
\renewcommand{\bibname}{References}
\addcontentsline{toc}{chapter}{References}
\begin{thebibliography}{99}
\bibitem{ReactNativeDocs}
React Native. (n.d.). React Native documentation. Retrieved from https://reactnative.dev/docs

\bibitem{ExpoDocs}
Expo. (n.d.). Expo documentation. Retrieved from https://docs.expo.dev

\bibitem{FirebaseDocs}
Firebase. (n.d.). Firebase documentation: Authentication, Firestore, and Storage. Retrieved from https://firebase.google.com/docs

\bibitem{PersonalFinanceApps}
Fernandes, D., Lynch, J. G., Jr., \& Netemeyer, R. G. (2014). Financial literacy, financial education, and downstream financial behaviors. \textit{Management Science, 60}(8), 1861--1883.

\bibitem{MobileUX}
Nielsen, J. (2012). Usability of mobile websites and applications. Nielsen Norman Group. Retrieved from https://www.nngroup.com

\bibitem{HCIForms}
Wroblewski, L. (2008). \textit{Web form design: Filling in the blanks}. Rosenfeld Media.

\bibitem{FormikDocs}
Formik. (n.d.). Formik documentation: Build forms in React, without the tears. Retrieved from https://formik.org

\bibitem{YupValidation}
Yup. (n.d.). Yup schema validation. Retrieved from https://github.com/jquense/yup

\bibitem{ReactNavigationDocs}
React Navigation. (n.d.). React Navigation documentation. Retrieved from https://reactnavigation.org/docs/getting-started

\bibitem{MobileFinance}
Prabhaker, P. R. (2020). Mobile wallet adoption and usage: A systematic literature review. \textit{Journal of Retailing and Consumer Services, 57}, 102207.

\bibitem{CloudStorage}
Mell, P., \& Grance, T. (2011). \textit{The NIST definition of cloud computing}. National Institute of Standards and Technology.

\bibitem{BudgetingApps}
Kantarci, B., \& Labatut, V. (2013). Classification of complex networks based on topological properties. \textit{IEEE Communications Surveys \& Tutorials, 15}(2), 729--754.

\end{thebibliography}



% Appendices
\appendix
\chapter{Screenshots and Results}

\section{Application Screens}
This section presents key screenshots from the SmartSpend mobile application, demonstrating the user interface and functionality across different screens.

\begin{figure}[H]
\centering
\includegraphics[width=0.95\textwidth]{output.jpeg}
\caption{SmartSpend Application Screens: Login, Dashboard, Add Expense, and Expense List views}
\label{fig:app-screenshots} 
\end{figure}

The application provides an intuitive interface with:
\begin{itemize}
    \item Clean authentication screens with form validation
    \item Dashboard with visual charts and monthly summaries
    \item Comprehensive expense entry form with all necessary fields
    \item Organized expense list with category indicators
    \item Responsive design across different device sizes
\end{itemize}

\section{Key Features Demonstrated}
\subsection{User Authentication}
The login and signup screens implement secure email/password authentication with real-time validation feedback and error handling.

\subsection{Dashboard Analytics}
The dashboard provides at-a-glance insights into spending patterns with pie charts showing category distributions and cards displaying monthly totals and reimbursable amounts.

\subsection{Expense Management}
Users can add detailed expense records with vendor information, amounts, dates, categories, and receipt attachments, all synchronized to the cloud in real-time.

\end{document}
